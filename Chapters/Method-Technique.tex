% Method-Technique

\chapter{Introduction}
\label{ch:introduction}

\chapter{Vector Autoregression}
\label{ch:vector-autoregression}

\textit{Vector autoregressive models (VAR)} is a multivariate
prediction technique where in order to predict a value for each
variable is necessary to take into account the past values of the
variable and of other variables involved in the model. In \textit{VAR}
all variables are interdependent, which means that all variables
depend on the other variables involved in the multivariate system.
This interdependent systems are known in econometric as
\textit{endogenous}.

\textit{VAR} is a generalization of \textit{Autoregressive Models
  (AR)} applied to a vector of time series. Lets analyze the
mathematical expression of a \textit{first order VAR}, or simply
\textit{VAR(1)} for a bivariate system defined in \autoref{eq:var}.

\begin{equation}
  \begin{aligned}
    \label{eq:var}
    y_{1,t} & = c_1 + \phi_{11,1} y_{1,t-1} + \phi_{12,1} y_{2,t-1} +
    e_{1,t} \\
    y_{2,t} & = c_2 + \phi_{21,1} y_{1,t-1} + \phi_{22,1} y_{2,t-1} +
    e_{2,t} 
  \end{aligned}
\end{equation}

In this particular example $y_{i, t}$ describes the variable $i$ in
the temporal instant $t$. This equations define the relationship
between the two variables involved. As the name suggests
\textit{VAR(1)} takes into account only the first lag of each
variable, where $c_i$ is a constant offset and $\phi_{ij,l}$ is the
influence of variable $y_j$ on variable $y_j$ in the $l$-th lag.

The generalized \textit{VAR(p)} expression would be
\autoref{eq:var-generalized}.

\begin{equation}
  \begin{aligned}
    \label{eq:var-generalized}
    \mathbf{y_t} = \mathbf{c} + \displaystyle\sum_{i=1}^p
    \pmb{\phi_i} \mathbf{y_{t-i}} + \pmb{\epsilon_t}
  \end{aligned}
\end{equation}


If the time series is stationary, we can predict their values by
directly fitting a \textit{VAR} to the data. On the other hand, if the
series is non-stationary we take differences to transform the time
series into a stationary one, and then, fit a \textit{VAR} to the
differenciated data. In both cases the model parameters are estimated
by \textit{OLS} per equation.

The predictions are generated recursively. \textit{VAR} generates a
prediction for each one of the variables involved in the model. Lets
take the \textit{VAR(1)} described in \autoref{eq:var}. \textit{VAR}
would estimate the parameters $\phi_{ij,l}$ and $c_i$ for $y_{i,t}$
such that $i \in \{1,2\}$ and $t = T$. Once the parameters has been
estimated by \textit{OLS} is possible to create the prediction
equations. \autoref{eq:var-prediction-1} describes the prediction
equation for $h=1$

\begin{equation}
  \begin{aligned}
    \label{eq:var-prediction-1}
    \hat{y}_{1,T+1|T} & = \hat{c}_1 + \hat{\phi}_{11,1} y_{1,T} +
    \hat{\phi}_{12,1} y_{2,T} \\ 
    \hat{y}_{2,T+1|T} & = \hat{c}_2 + \hat{\phi}_{21,1} y_{1,T} +
    \hat{\phi}_{22,1} y_{2,T}
  \end{aligned}
\end{equation}

\autoref{eq:var-prediction-1} is very similar to \autoref{eq:var}
except that the parameters has been replaced by its estimated values
and the error has been replaced by zero.

If the horizon we want to use is $h = 2$, the equation would be
\autoref{eq:var-prediction-2} where the values for past values of the
variable are replaced by predictions of those values.

\begin{equation}
  \begin{aligned}
    \label{eq:var-prediction-2}
    \hat{y}_{1,T+2|T} & = \hat{c}_1 + \hat{\phi}_{11,1} \hat{y}_{1,T+1} +
    \hat{\phi}_{12,1} \hat{y}_{2,T+1} \\ 
    \hat{y}_{2,T+2|T} & = \hat{c}_2 + \hat{\phi}_{21,1} \hat{y}_{1,T+1} +
    \hat{\phi}_{22,1} \hat{y}_{2,T+1}
  \end{aligned}
\end{equation}

In \textit{VAR} there are two parameters needed for fitting a model.
One is the number of variables, denoted by $K$, and the other one is
the number of lags or $p$. The number of parameters to be estimated in
a \textit{VAR} is $K + p K^2$ or $1 + p K$ per equation. For
\autoref{eq:var} there are two variables, i.e. $K=2$, and only the
first lag is used, i.e. $p=1$, that makes a total of parameters to be
estimated of $2 + 1 \times 2^2 = 6$.

\chapter{Recurrent Neural Networks}
\label{ch:recurrent-neural-networks}




%---------------------------------------------------------------------
%---------------------------------------------------------------------
%---------------------------------------------------------------------

%\enlargethispage{2cm}

%------------------------------------------------

%%% Local Variables:
%%% mode: latex
%%% TeX-master: "../main"
%%% End:
