% State Of The Art

\chapter{Bitcoin Trading State Of The Art} % Chapter title

\label{ch:bitcoin-trading-state-of-the-art}

%----------------------------------------------------------------------

\section[\cite{madan_automated_2014}] {Automated Bitcoin Trading
  via Machine Learning Algorithms}

\label{sec:automated-bitcoin-trading-via-machine-learning-algorithms}

\cite{madan_automated_2014} forecast the price of BTC for periods
of 24h (phase one) and 10 minutes/seconds (phase two), obtaining an
accuracy of $98.7\%$
for phase one and $50\% - 55\%$
for phase two. The data set for \textit{first phase} consists of about
16 features (shown in \autoref{tab:bitcoin-features-madan}) relating
to the BTC price and payment network over the course of five years,
recorded daily. The features were selected manually based on research
of the significance to the problem of forecasting. For the
\textit{second phase} the data consisted of 10-second and 10-minute
interval BTC price.

In \textit{first phase} they used Binomial GLM, SVM and Random Forest.
Binomial GLM shows higher precision \& accuracy that the other two
algorithms.

For the \textit{second phase} they used only Binomial GLM and Random
Forest. Performing 10 second and 10 minute prediction with Binomial
GLM and comparing with 10 minute prediction for Random Forest. It
turns out that Random Forest shows the higher accuracy \& precision
than GLM in both 10 seconds and 10 minutes interval prediction.

\begin{table}
  \myfloatalign
  \begin{tabularx}{\textwidth}{XX} 
    \toprule
    \tableheadline{feature} & \tableheadline{definition} \\ 
    \midrule
    Average Confirmation Time & Ave. time to accept transaction in
                                block \\
    Block size & Average block size in MB \\
    Cost per transaction percent & Miners revenue divided by the
                                   number of transactions \\
    Difficulty & How difficult it is to find a new block \\
    Estimated Transaction Volume & Total output volume without change
                                   from value \\
    Hash Rate & Bitcoin network giga hashes per second \\
    Market Capitalization & Number of Bitcoins in circulation * the
                            marker price \\
    Miners Revenue & (number of BTC mined/day * market price) +
                     transaction fees \\
    Number of Orphaned Blocks & Number of blocks mined/day not off
                                blockchain \\
    Number of TXN per block & Average number of transactions per block
    \\ 
    Number of TXN & Total number of unique Bitcoin transactions per
                    day \\
    Number of unique addresses & Number of unique Bitcoin addresses
                                 used per day \\
    Total Bitcoins & Historical total Number of Bitcoins mined \\
    TXN Fees Total & BTC value of transactions fees miners earn/day \\
    Trade Volume & USD trade volume from the top exchanges \\
    Transactions to trade ratio & Relationship of BTC transaction
                                  volume and USD volume \\
    \bottomrule
  \end{tabularx}
  \caption{Bitcoin features selected by
    \cite{madan_automated_2014}}
  \label{tab:bitcoin-features-madan}
\end{table}

\section[\cite{georgoula_using_2015}]{Using Time-Series and Sentiment
  Analysis to detect the Determinants of Bitcoin Prices}

\label{sec:using-time-series-and-sentiment-analysis}

\cite{georgoula_using_2015} used time-series analysis to study the
relationship between Bitcoin prices and fundamental economic
variables, technological factors and measurements of collective mood
derived from Twitter feeds. They conclude that Bitcoin price is
positively associated with the number of Bitcoins in circulation and
negatively associated with the Standard and Poor's 500 stock market
index.

\begin{table}
  \myfloatalign
  \begin{tabularx}{\textwidth}{cX} 
    \toprule
    \tableheadline{Name of variable} & \tableheadline{Description} \\ 
    \midrule
    \textit{bcp} & Bitstamp daily closing price \\
    \textit{totbc} & Total daily number of Bitcoins in circulation \\
    \textit{ntran} & Total daily number of unique Bitcoin transactions
    \\
    \textit{bcdde} & Bitcoin days destroyed for any given transaction
    \\
    \textit{exrate} & Daily exchange rate between the USD and the
    euro(\$/€) \\
    \textit{sp} & Standar \& Poor's 500 stock market daily index \\
    \textit{hash} & Processing power required for the secure operation
    of Bitcoin network (in billions of hashes per second) \\
    \textit{wiki} & Daily number of Bitcoin search queries on
    Wikipedia \\
    \textit{google} & Daily number of Bitcoin search queries in Google
    \\
    \textit{ntweets} & Daily number of Twitter posts related to
    Bitcoins \\
    \textit{sent} & Daily sentiment ration of Twitter posts related to
    Bitcoins \\
    \bottomrule
  \end{tabularx}
  \caption{Bitcoin features selected by
    \cite{georgoula_using_2015}}
  \label{tab:bitcoin-features-georgoula-2015}
\end{table}

They used time-series to forecast the Bitcoin price and SVM to
classify the Tweet feeds sentiment.

As seen by ordinary least square regression model obtained, the paper
suggests the next conclusions:

\begin{itemize}
\item \textit{logwiki} (representing the degree of public recognition
  or interest in Bitcoins) and \textit{loghash} (measuring the mining
  difficulty) have a positive impact on Bitcoin price 
\item The exchange rate between the USD and the euro relationship is
  negative 
\item The sentiment ration of Twitter users, and the number of
  Wikipedia search queries positively affects the price of Bitcoins
\item The stock of Bitcoins has a positive long-run impact on its
  price 
\item The Standard and Poor's 500 index was found to have a negative
  impact on Bitcoin prices in the long run
\end{itemize}

%---------------------------------------------------------------------

\section[\cite{kristoufek_what_2015}]{What are the main drivers of the
  Bitcoin price? Evidence from Wavelet Coherence Analysis}
\label{sec:where-are-the-main-drivers-of-the-bitcoin-price}

\cite{kristoufek_what_2015} examines in this paper the potential
drivers of Bitcoin price by means of wavelet coherence analysis of
different variables. This paper doesn't try to perform a forecast,
just select the variables that are \textit{``drive''} the price of
BTC.

The conclusions after the analysis are:
\begin{itemize}
\item The \textit{Trade Exchange Ratio} has a strong , but not
  statistically significant at the 5\% level, relationship at high
  scales. The price and the ratio are negatively correlated in the
  long term
\item Bitcoin appreciates in the long run if it is used more for
  trade, i.e., non-exchange transactions, and hte increasing price
  boosts the exchange transactions in the short run
\item The money supply works as a standard supply, so that its
  increase leads to a price decrease
\item For the trade transactions, the increasing usage of bitcois in
  real transactions leads to an appreciation of the bitcoin in the
  long run
\item For the trade volume the relationship changes in time to offer
  any strong conclusion.
\item Both measures of the mining difficulty (\textit{Hash} and
  \textit{Difficulty}) are positively correlated with the price in the
  long run
\item In the short run, Bitcoin price and both \textit{Hash} and
  \textit{Difficulty} relationship becomes negative
\item The interest (\textit{Google} and \textit{Wikipedia}) in Bitcoin
  appears to have an asymmetric effect during the bubble formation and
  its bursting. During the bubble formation, interest boosts the
  prices further, and during the bursting, it pushes them lower.
\item For the \textit{FSI}, apart from the Cypriot crisis, there are
  no longer-term time intervals during which the correlations are both
  statistically significant and realiable
\item Gold price apparently has no relationship with BTC price
\item Although the USD and CNY (Chinese renminbi) markets are tightly
  connected, there is no clear evidence that the Chinese market
  influences the USD market.
\end{itemize}

\begin{table}
  \myfloatalign
  \begin{tabularx}{\textwidth}{cX} 
    \toprule
    \tableheadline{Name of variable} & \tableheadline{Description} \\
    \midrule \textit{BPI} & Bitcoin price index, is an exchange rate
    between USD and BTC. \\
    \textit{Total BTC} & Total Bitcoins in circulation \\
    \textit{NumTrans} & Number of transactions excluding exchange
    transactions \\
    \textit{Out Val} & Estimated output volume \\
    \textit{Trade Exchange Ratio} & Trade volume vs. transaction volume ratio \\
    \textit{Hash} & Hash rate \\
    \textit{Difficulty} & Current computation power of the system
    measured in hashes \\
    \textit{Exchange} & Time series of exchange rates between BTC and
    various currencies \\
    \textit{Google} & Weekly number of Bitcoin search queries in
    Google \\
    \textit{Wikipedia} & Daily number of Wikipedia search queries
    related to Bitcoins \\
    \textit{FSI} & Financial Stress Index provided by the Federal
    Reserve Bank of Cleveland \\
    \textit{Gold price} & Gold price for a troy ounce in Swiss francs (CHF) \\
    \bottomrule
  \end{tabularx}
  \caption{Bitcoin features selected by
    \cite{kristoufek_what_2015}}
  \label{tab:bitcoin-features-kristoufek}
\end{table}

%---------------------------------------------------------------------

\section[\cite{garcia_social_2015}]{Social signals and algorithmic
  trading of Bitcoin}
\label{sec:social-signals-and-algorithmic-trading-of-bitcoin}

This paper by \cite{garcia_social_2015} provides a consistent approach
that integrates various datasources in the design of algorithmic
traders. They applied multidimensional model of vector auto-regression
to design different types of trading algorithms. The analysis
performed reveals that increases in opinion polarization and exchange
volume precede rising Bitcoin prices, and that emotional valence
precedes opinion polarization and rising exchange volumes. They
choosed a number of Bitcoin related variables based on the work of
other researchers. They can be seen in
\autoref{tab:bitcoin-features-garcia}.

\begin{table}
  \myfloatalign
  \begin{tabularx}{\textwidth}{cX} 
    \toprule
    \tableheadline{Name of variable} & \tableheadline{Description} \\
    \midrule
    $P(t)$ & Price \\
    $Ret(t)$ & Return \\
    $FX_{Vol}(t)$ & Trading volume \\
    $BC_{Tra}(t)$ & Transaction volume in the Block Chain \\
    $Dwn(t)$ & Amount of downloads of the most important Bitcoin client \\
    $S(t)$ & Level of search volume in Google for the term ``bitcoin'' \\
    $T_N(t)$ & Amount of tweets containing Bitcoin-related terms \\
    $T_{Val}(t)$ & Emotional valence \\
    $T_{Pol}(t)$ & Opinion polarization expressed in the tweets \\
    \bottomrule
  \end{tabularx}
  \caption{Bitcoin features selected by
    \cite{garcia_social_2015}}
  \label{tab:bitcoin-features-garcia}
\end{table}

%---------------------------------------------------------------------
%---------------------------------------------------------------------
%---------------------------------------------------------------------


%\enlargethispage{2cm}

%------------------------------------------------

%%% Local Variables:
%%% mode: latex
%%% TeX-master: "../main"
%%% End:
