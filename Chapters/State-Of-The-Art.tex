% State-Of-The-Art

\chapter{Bitcoin Trading State-Of-The-Art} % Chapter title

\label{ch:bitcoin-trading-state-of-the-art}

%----------------------------------------------------------------------

\section[\cite{madan_automated_2014}] {Automated Bitcoin Trading
  via Machine Learning Algorithms}

\label{sec:automated-bitcoin-trading-via-machine-learning-algorithms}

\cite{madan_automated_2014} forecast the price of BTC for periods
of 24h (phase one) and 10 minutes/seconds (phase two), obtaining an
accuracy of $98.7\%$
for phase one and $50\% - 55\%$
for phase two. The data set for \textit{first phase} consists of about
16 features (shown in \autoref{tab:bitcoin-features-madan}) relating
to the BTC price and payment network over the course of five years,
recorded daily. The features were selected manually based on research
of the significance to the problem of forecasting. For the
\textit{second phase} the data consisted of 10-second and 10-minute
interval BTC price.

In \textit{first phase} they used Binomial GLM, SVM and Random Forest.
Binomial GLM shows higher precision \& accuracy that the other two
algorithms.

For the \textit{second phase} they used only Binomial GLM and Random
Forest. Performing 10 second and 10 minute prediction with Binomial
GLM and comparing with 10 minute prediction for Random Forest. It
turns out that Random Forest shows the higher accuracy \& precision
than GLM in both 10 seconds and 10 minutes interval prediction.

\begin{table}
  \scriptsize
  \myfloatalign
  \begin{tabularx}{\textwidth}{XX} 
    \toprule
    \tableheadline{feature} & \tableheadline{definition} \\ 
    \midrule
    Average Confirmation Time & Ave. time to accept transaction in
                                block \\
    Block size & Average block size in MB \\
    Cost per transaction percent & Miners revenue divided by the
                                   number of transactions \\
    Difficulty & How difficult it is to find a new block \\
    Estimated Transaction Volume & Total output volume without change
                                   from value \\
    Hash Rate & Bitcoin network giga hashes per second \\
    Market Capitalization & Number of Bitcoins in circulation * the
                            marker price \\
    Miners Revenue & (number of BTC mined/day * market price) +
                     transaction fees \\
    Number of Orphaned Blocks & Number of blocks mined/day not off
                                blockchain \\
    Number of TXN per block & Average number of transactions per block
    \\ 
    Number of TXN & Total number of unique Bitcoin transactions per
                    day \\
    Number of unique addresses & Number of unique Bitcoin addresses
                                 used per day \\
    Total Bitcoins & Historical total Number of Bitcoins mined \\
    TXN Fees Total & BTC value of transactions fees miners earn/day \\
    Trade Volume & USD trade volume from the top exchanges \\
    Transactions to trade ratio & Relationship of BTC transaction
                                  volume and USD volume \\
    \bottomrule
  \end{tabularx}
  \caption{Bitcoin features selected by
    \cite{madan_automated_2014}}
  \label{tab:bitcoin-features-madan}
\end{table}

%---------------------------------------------------------------------

\section[\cite{georgoula_using_2015}]{Using Time-Series and Sentiment
  Analysis to detect the Determinants of Bitcoin Prices}

\label{sec:using-time-series-and-sentiment-analysis}

\cite{georgoula_using_2015} used time-series analysis to study the
relationship between Bitcoin prices and fundamental economic
variables, technological factors and measurements of collective mood
derived from Twitter feeds. They conclude that Bitcoin price is
positively associated with the number of Bitcoins in circulation and
negatively associated with the Standard and Poor's 500 stock market
index.

\begin{table}
  \scriptsize
  \myfloatalign
  \begin{tabularx}{\textwidth}{cX} 
    \toprule
    \tableheadline{Name of variable} & \tableheadline{Description} \\ 
    \midrule
    \textit{bcp} & Bitstamp daily closing price \\
    \textit{totbc} & Total daily number of Bitcoins in circulation \\
    \textit{ntran} & Total daily number of unique Bitcoin transactions
    \\
    \textit{bcdde} & Bitcoin days destroyed for any given transaction
    \\
    \textit{exrate} & Daily exchange rate between the USD and the
    euro(\$/€) \\
    \textit{sp} & Standar \& Poor's 500 stock market daily index \\
    \textit{hash} & Processing power required for the secure operation
    of Bitcoin network (in billions of hashes per second) \\
    \textit{wiki} & Daily number of Bitcoin search queries on
    Wikipedia \\
    \textit{google} & Daily number of Bitcoin search queries in Google
    \\
    \textit{ntweets} & Daily number of Twitter posts related to
    Bitcoins \\
    \textit{sent} & Daily sentiment ration of Twitter posts related to
    Bitcoins \\
    \bottomrule
  \end{tabularx}
  \caption{Bitcoin features selected by
    \cite{georgoula_using_2015}}
  \label{tab:bitcoin-features-georgoula-2015}
\end{table}

They used time-series to forecast the Bitcoin price and SVM to
classify the Tweet feeds sentiment.

As seen by ordinary least square regression model obtained, the paper
suggests the next conclusions:

\begin{itemize}
\item \textit{logwiki} (representing the degree of public recognition
  or interest in Bitcoins) and \textit{loghash} (measuring the mining
  difficulty) have a positive impact on Bitcoin price 
\item The exchange rate between the USD and the euro relationship is
  negative 
\item The sentiment ration of Twitter users, and the number of
  Wikipedia search queries positively affects the price of Bitcoins
\item The stock of Bitcoins has a positive long-run impact on its
  price 
\item The Standard and Poor's 500 index was found to have a negative
  impact on Bitcoin prices in the long run
\end{itemize}

%---------------------------------------------------------------------

\section[\cite{kristoufek_what_2015}]{What are the main drivers of the
  Bitcoin price? Evidence from Wavelet Coherence Analysis}
\label{sec:where-are-the-main-drivers-of-the-bitcoin-price}

\cite{kristoufek_what_2015} examines in this paper the potential
drivers of Bitcoin price by means of wavelet coherence analysis of
different variables. This paper doesn't try to perform a forecast,
just select the variables that are \textit{``drive''} the price of
BTC.

The conclusions after the analysis are:
\begin{itemize}
\item The \textit{Trade Exchange Ratio} has a strong , but not
  statistically significant at the 5\% level, relationship at high
  scales. The price and the ratio are negatively correlated in the
  long term
\item Bitcoin appreciates in the long run if it is used more for
  trade, i.e., non-exchange transactions, and hte increasing price
  boosts the exchange transactions in the short run
\item The money supply works as a standard supply, so that its
  increase leads to a price decrease
\item For the trade transactions, the increasing usage of bitcois in
  real transactions leads to an appreciation of the bitcoin in the
  long run
\item For the trade volume the relationship changes in time to offer
  any strong conclusion.
\item Both measures of the mining difficulty (\textit{Hash} and
  \textit{Difficulty}) are positively correlated with the price in the
  long run
\item In the short run, Bitcoin price and both \textit{Hash} and
  \textit{Difficulty} relationship becomes negative
\item The interest (\textit{Google} and \textit{Wikipedia}) in Bitcoin
  appears to have an asymmetric effect during the bubble formation and
  its bursting. During the bubble formation, interest boosts the
  prices further, and during the bursting, it pushes them lower.
\item For the \textit{FSI}, apart from the Cypriot crisis, there are
  no longer-term time intervals during which the correlations are both
  statistically significant and realiable
\item Gold price apparently has no relationship with BTC price
\item Although the USD and CNY (Chinese renminbi) markets are tightly
  connected, there is no clear evidence that the Chinese market
  influences the USD market.
\end{itemize}

\begin{table}
  \scriptsize
  \myfloatalign
  \begin{tabularx}{\textwidth}{cX} 
    \toprule
    \tableheadline{Name of variable} & \tableheadline{Description} \\
    \midrule \textit{BPI} & Bitcoin price index, is an exchange rate
    between USD and BTC. \\
    \textit{Total BTC} & Total Bitcoins in circulation \\
    \textit{NumTrans} & Number of transactions excluding exchange
    transactions \\
    \textit{Out Val} & Estimated output volume \\
    \textit{Trade Exchange Ratio} & Trade volume vs. transaction volume ratio \\
    \textit{Hash} & Hash rate \\
    \textit{Difficulty} & Current computation power of the system
    measured in hashes \\
    \textit{Exchange} & Time series of exchange rates between BTC and
    various currencies \\
    \textit{Google} & Weekly number of Bitcoin search queries in
    Google \\
    \textit{Wikipedia} & Daily number of Wikipedia search queries
    related to Bitcoins \\
    \textit{FSI} & Financial Stress Index provided by the Federal
    Reserve Bank of Cleveland \\
    \textit{Gold price} & Gold price for a troy ounce in Swiss francs (CHF) \\
    \bottomrule
  \end{tabularx}
  \caption{Bitcoin features selected by
    \cite{kristoufek_what_2015}}
  \label{tab:bitcoin-features-kristoufek}
\end{table}

%---------------------------------------------------------------------

\section[\cite{garcia_social_2015}]{Social signals and algorithmic
  trading of Bitcoin}
\label{sec:social-signals-and-algorithmic-trading-of-bitcoin}

This paper by \cite{garcia_social_2015} provides a consistent approach
that integrates various datasources in the design of algorithmic
traders. They applied multidimensional model of vector auto-regression
to design different types of trading algorithms. The analysis
performed reveals that increases in opinion polarization and exchange
volume precede rising Bitcoin prices, and that emotional valence
precedes opinion polarization and rising exchange volumes. They
choosed a number of Bitcoin related variables based on the work of
other researchers. They can be seen in
\autoref{tab:bitcoin-features-garcia}.

\begin{table}
  \scriptsize
  \myfloatalign
  \begin{tabularx}{\textwidth}{cX} 
    \toprule
    \tableheadline{Name of variable} & \tableheadline{Description} \\
    \midrule
    $P(t)$ & Price \\
    $Ret(t)$ & Return \\
    $FX_{Vol}(t)$ & Trading volume \\
    $BC_{Tra}(t)$ & Transaction volume in the Block Chain \\
    $Dwn(t)$ & Amount of downloads of the most important Bitcoin client \\
    $S(t)$ & Level of search volume in Google for the term ``bitcoin'' \\
    $T_N(t)$ & Amount of tweets containing Bitcoin-related terms \\
    $T_{Val}(t)$ & Emotional valence \\
    $T_{Pol}(t)$ & Opinion polarization expressed in the tweets \\
    \bottomrule
  \end{tabularx}
  \caption{Bitcoin features selected by
    \cite{garcia_social_2015}}
  \label{tab:bitcoin-features-garcia}
\end{table}

%---------------------------------------------------------------------
%---------------------------------------------------------------------

\chapter{Financial forecasting State-Of-The-Art}
\label{chap:financial-forecasting-state-of-the-art}

%---------------------------------------------------------------------

\section[\cite{bahrammirzaee2010comparative}]{A comparative survey of
  artificial intelligence applications in finance: artificial neural
  networks, expert system and hybrid intelligent systems}
\label{sec:a-comparative-survey-of-artificial-intelligence-applications}

In this article by \cite{bahrammirzaee2010comparative} several
forecasting and classification techniques are compared to three
important AI techniques, namely, artificial neural networks, expert
systems and hybrid intelligence systems. For the purpose of this
thesis we are going to show the results related to ANNs in
\autoref{tab:ann-compared-to-others}.

\begin{table}[htbp]
  \tiny
  \myfloatalign
  \begin{tabularx}{\textwidth}{XXXX} 
    \toprule
    \tableheadline{Author/s} & \tableheadline{Domain} & \tableheadline{Method compared with} & \tableheadline{Result} \\
    \midrule
    \cite{kim1998graded} & Stock market index forecasting & Reined
    PNN, APN, BPNN, RNN and CBR & CBR performs better \\ 
    \cite{nasir2001selecting} & Bankruptcy prediction & Fully
    connected BPNN versus interconnected BPNN & Interconnected BPNN
    performs better \\ 
    \cite{zhang2001time} & Exchange rate prediction & Ensemble NNs
    with single NN & Ensemble NNs perform better \\ 
    \cite{wang2004prediction} & Stock market prediction & Gradient
    descent with adaptive learning rate BP, gradient descent with
    momentum \& adaptive learning rate BP, LM BP, BFGS BP,
    quasi-Newton and RPROB BP & LM BPNN performs better \\ 
    \cite{zhang2004neural} & Earning per share forecasting & BPNN with
    ARIMA and LR & BPNN perform better \\ 
    \cite{majhi2009efficient} & Exchange rate prediction & CFLANN with
    FLANN and standard LMS based forecasting model & CFLANN perform
    better \\ 
    \cite{tsai2009feature} & Feature selection in bankruptcy & $T$
    test, correlation matrix, stepwise regression, PCA and FA for
    feature selection in BPNN & $T$ test method is best method for
    BPNN future selection \\  
    \cite{leu2009distance} & Exchange rate forecasting &
    Distance-based fuzzy time series with random walk model and ANN &
    Distance-based fuzzy time series perform better \\ 
    \cite{ghazali2009non} & Financial time series prediction & DRPNN
    with Pi-Sigma NN and the ridge polynomial NN & DRPNN performs
    better \\ 
    \cite{cao2009neural} & Earning per share forecasting & BPNN
    (suggests \cite{zhang2004neural}) with GA) & GA performs better \\ 
    \bottomrule
  \end{tabularx}
  \caption{Comparisons between ANN model/s in financial prediction and
    planning in \cite{bahrammirzaee2010comparative}}
  \label{tab:ann-compared-to-others}
\end{table}

I want to include the next quote from the article because describes
very well how NNs behave in a financial scenario: \textit{``Regarding
  NNs, the empirical results of these comparative studies indicate
  that the success of NNs in financial domain, especially in financial
  prediction and planning, is very encouraging. (However, while NNs
  often outperform the more traditional and statistical approaches but
  this is not always the case. There are some studies in which other
  traditional methods, or intelligent approach outperforms NNs.) This
  success is due to some unique characteristics of NNs in financial
  market like their numeric nature, no requirement to any data
  distribution assumptions (for inputs) and model estimators and
  finally, their capability to update the data. Despite this success,
  this paper could not conclude that NNs are very accurate techniques
  in financial market because at the first, among these studies, BPNN
  is the most popular NN training technique. However, BPNN suffers
  from the potential problems. One of the problems of BPNN is local
  minimum convergence. Because the gradient search process proceeds in
  a point-to-point mode and is intrinsically local in scope,
  convergence to local rather than global minima is a very
  possibility. Also BP training method is very slow and takes too much
  time to converge. Besides, it can over fit the training data.
  Secondly, it is difficult, if not impossible, to determine the
  proper size and structure of a neural net to solve a given problem.
  Therefore, the architectural parameters are generally designed by
  researchers and via trial and errors and since these parameters
  determine outputs of NNs, their accuracy and performance are subject
  to numerous errors. Also sometimes NNs are incapable to recognize
  patterns and establish relevant relationships between various
  factors, which are important reasons to reduce their performance.
  Finally, NNs learn based on past occurrences, which may not be
  repeated, especially in financial markets and in current financial
  crisis.''}

%---------------------------------------------------------------------

\section[\cite{atsalakis2009surveying}]{Surveying stock market
  forecasting techniques - Part II: Soft computing methods}
\label{sec:surveing-stock-market-forecasting-techniques:part2}

This paper enumerates over a 100 scientific articles applied to stock
market forecasting with a focus on neural networks and neuro-fuzzy
techniques. \cite{atsalakis2009surveying} enumerate the modeling
techniques used to forecast, the stock markets whose information are
used (not very useful for this thesis because we are interested in
Bitcoin where the information source is clear), the input variables,
which we summarize in \autoref{tab:input-variables-atsalakis2009},
whether they perform data processing, sample sizes, network layers,
membership functions, validation set and training method. They also
show the methods used to compare with for each article without giving
the conclusions made in each article. Another very interesting
information are the \textit{model performance measures} where the
measures are classified as statistical and non-statistical measures:
\textit{``Statistical measures include the root mean square error
  (RMSE), the mean absolute error (MAE) and the mean squared
  prediction error (MSPE), statistical indicators like the
  autocorrelation, the correlation coefficient, the mean absolute
  deviation, the squared correlation and the standard deviation. [...]
  Non-statistical performance measures include measures that are
  related with the economical side of the forecast. The most common
  used performance measure is the so called Hit Rate that measures the
  percentage of correct predictions of the model. Another two measures
  that deal with the profitability of the model are the annual rate of
  return and the average annual profit of the model''}

\begin{table}[htbp]
  \scriptsize
  \myfloatalign
  \begin{tabularx}{\textwidth}{X} 
    \toprule
    \tableheadline{Variable set} \\
    \midrule
    Eight input variables (NASDAQ and 7 indexes) \\ 
    Opening, lowest, highest and closing index price \\ 
    Index values of five previous days\\ 
    Ten variables (five Technical Analysis factors, price of last 5
    days) \\
    Eight fundamental analysis indicators \\
    Change of stock prices in last two sessions \\
    Three technical analysis indexes \\
    Volume, opening, lowest, highest and closing index price \\
    Fifteen technical analysis factors \\ Changes in stock prices \\
    Volume ratio, relative strength index, rate of change, slow \%D \\
    Daily closing value \\
    Technical Analysis variables \\
    FTSE index, exchange rate, futures market etc \\
    Beta, cap, b/m \\
    Seven economical indicators \\
    \bottomrule
  \end{tabularx}
  \caption{16 example sets of variables used by different articles extracted 
    from \cite{atsalakis2009surveying}}
  \label{tab:input-variables-atsalakis2009}
\end{table}

As a conclusion, they mention that: \textit{``The observation is that
  neural networks and neuro-fuzzy models are suitable for stock market
  forecasting. Experiments demonstrate that soft computing techniques
  outperform conventional models in most cases. They return better
  results as trading systems and higher forecasting accuracy. However,
  difficulties arise when defining the structure of the model (the
  hidden layers the neurons etc.). For the time being, the structure
  of the model is a matter of trial and error procedures.``}

%---------------------------------------------------------------------

\section[\cite{carpinteiro2012forecasting}]{Forecasting models for
  prediction in time series}
\label{sec:forecasting-models-for-prediction-in-time-series}

\cite{carpinteiro2012forecasting} give a comparison of three
forecasting models: a multilayer perceptron, a support vector machine,
and a hierarchical model. After performing six experiments running
through time series of the stock market fund on Bank of Brasil, they
concluded that the performance of HM is better than SVM and MLP. As a
possible explanation, they suggest that the superiority of HM can be
due to its hierarchical architecture.

%---------------------------------------------------------------------

\section[\cite{krollner2010financial}]{Financial Time Series
  Forecasting with Machine Learning Techniques: A Survey}
\label{sec:financial-time-series-forecasting-with-machine-learning}

\cite{krollner2010financial} compare 46 forecasting studies. They
divide the papers into artificial neural network (ANN) based and
evolutionary \& optimisation based techniques. Related to this
division they show the amount of studies using each technique, which
we reproduce in \autoref{tab:reviewed-papers-machine-learning}.

\begin{table}[htbp]
  \scriptsize
  \myfloatalign
  \begin{tabularx}{\textwidth}{Xc} 
    \toprule
    \tableheadline{Technology} & \tableheadline{Number} \\ 
    \midrule
    ANN based & 21 \\
    Evolutionary \& optimisation techniques & 4 \\
    Multiple / hybrid & 15 \\
    Other & 6 \\
    \bottomrule
  \end{tabularx}
  \caption{Reviewed papers classified by machine learning technique by
    \cite{krollner2010financial}} 
  \label{tab:reviewed-papers-machine-learning}
\end{table}

When it comes to the prediction periods, they categorised it in one
day, one week, and one month ahead predictions, shown in
\autoref{tab:reviewed-classified-time-frame}

\begin{table}[htbp]
  \scriptsize
  \myfloatalign
  \begin{tabularx}{\textwidth}{Xc} 
    \toprule
    \tableheadline{Time-frame} & \tableheadline{Number} \\ 
    \midrule
    Day & 31 \\
    Week & 3 \\
    Month & 3 \\
    Multiple / Other & 9 \\
    \bottomrule
  \end{tabularx}
  \caption{Reviewed papers classified by forecasting time-frame by
    \cite{krollner2010financial}} 
  \label{tab:reviewed-classified-time-frame} 
\end{table}

The 75\% of papers reviewed use some sort of lagged index data as
input variables. Most commonly used parameters are daily opening,
high, low and close prices. Also several technical indicators are
used, been the more important: Simple moving average (SMA),
exponential moving average (EMA), relative strength index (RSI), rate
of change (ROC), moving average convergence / divergence (MACD) and
William's oscillator and average true range (ATR).
\autoref{tab:reviewed-papers-by-input-variables} displays the number
of studies using the different input variables.

\begin{table}[htbp]
  \scriptsize
  \myfloatalign
  \begin{tabularx}{\textwidth}{Xc} 
    \toprule
    \tableheadline{Input} & \tableheadline{Number} \\ 
    \midrule
    Lagged Index Data & 35 \\
    Trading volume & 4 \\
    Technical Indicators & 13 \\
    Oil Price & 4 \\
    S\&P 500 / NASDAQ /Dow Jones (non US studies) & 4 \\
    Unemployment Rate & 1 \\
    Money Supply & 3 \\
    Exchange Rates & 3 \\
    Gold Price & 3 \\
    Short \& Long Term Interest Rates & 6 \\
    Others & 10 \\
    \bottomrule
  \end{tabularx}
  \caption{Reviewed papers classified by input variables by
    \cite{krollner2010financial}} 
  \label{tab:reviewed-papers-by-input-variables}
\end{table}

The evaluation methods are diverse, and the categories used are buy \&
hold, random walk, statistical techniques, other machine learning
techniques , and no benchmark model. The relationship between
categories and the studies is reproduced in
\autoref{tab:papers-by-evaluation-models}.

\begin{table}[htbp]
  \scriptsize
  \myfloatalign
  \begin{tabularx}{\textwidth}{Xc} 
    \toprule
    \tableheadline{Evaluation Model} & \tableheadline{Number} \\ 
    \midrule
    Buy \& Hold & 9 \\
    Random Walk & 6 \\
    Statistical Techniques & 18 \\
    Other Machine Learning Techniques & 28 \\
    No Benchmark Model & 7 \\
    \bottomrule
  \end{tabularx}
  \caption{Reviewed papers classified by machine learning technique by
    \cite{krollner2010financial}} 
  \label{tab:papers-by-evaluation-models}
\end{table}

An interesting quote made by \cite{krollner2010financial} is:
\textit{``Over 80\% of the papers report that their model outperformed
  the benchmark model. However, most analysed studies do not consider
  real world constraints like trading costs and splippage.``}

As one of the conclusions of the review they state that artificial
neural networks are the dominant techniques for the field of financial
forecasting.

%---------------------------------------------------------------------

\section[\cite{beiranvand_comparative_2012}]{A Comparative Survey of
  Three AI Techniques (NN, PSO, and GA) in Financial Domain}
\label{sec:a-comparative-survey-of-three-ai-techniques-nn-pso-and-ga}

This paper reviews ANNs, GAs and PSOs, and compare their application
in financial domain, dividing, as previous papers did, in three
subdomains: portfolio managemente, financial forecasting and planning,
and credit evaluation. ANNs prove to be a good method when compared to
traditional techniques, and to note that, they say: \textit{``In 2008,
  two Artificial NNs were developed by Angelini et al. . One of them
  was based on a regular feed-forward model and the other one had
  specific purpose architecture. The performance of the system was
  assessed using real world data, acquired from small corporations in
  Italia. They proved that if data analysis and preprocessing are done
  carefully, and suitable training is carried out, then Artificial NNs
  will demonstrate a powerful performance in estimating and learning
  the default direction of a borrower.''}, as shown in
\autoref{tab:brief-ann-comparison}.

\begin{table}[htbp]
  \scriptsize
  \myfloatalign
  \begin{tabularx}{\textwidth}{XXXX} 
    \toprule
    \tableheadline{Domain} & \tableheadline{Author(s)} &
    \tableheadline{Approaches compared} & \tableheadline{Conclusion} \\ 
    \midrule
    Credit evaluation & \cite{malhotra2003evaluating} &
    Back-propagation NN compared with MDA & Artificial NN has a better
    performance \\
    \midrule
    Credit scoring & \cite{abdou2008neural} & PNN and MLP compared
    with DA and LR & PNN and MLP have a better performance \\
    \midrule
    Credit mearument & \cite{bennell2006modelling} & Back-Propagation
    NN with OPM & Artificial NN has a better performance \\
    \midrule
    Bond scoring & \cite{kaplan1979statistical} & Back-Propagation NN
    compared with LR & Artificial NN has a better performance \\
    \midrule
    Credit scoring & \cite{baesens2003benchmarking} & BPNN compared
    with MDA, LR, and LS-SVMs & LS-SVMs and BPNN have better
    performance \\
    \bottomrule
  \end{tabularx}
  \caption{Comparison between ANNs and traditional techniques by
    \cite{beiranvand_comparative_2012}}
  \label{tab:brief-ann-comparison}
\end{table}

When applied to portfolio management, ANNs has better performance, and
is particularly true for back-propagation NNs.
\autoref{tab:ann-comparison-portfolio-manage} shows a brief comparison
made by \cite{beiranvand_comparative_2012}. The same goes for
forecasting and planning, where
\autoref{tab:ann-comparison-forecasting} shows the comparison with
other methods.

\begin{table}[htbp]
  \scriptsize
  \myfloatalign
  \begin{tabularx}{\textwidth}{XXXX} 
    \toprule
    \tableheadline{Domain} & \tableheadline{Author(s)} &
    \tableheadline{Approaches compared} & \tableheadline{Conclusion} \\ 
    \midrule
    Mortages choice decision & \cite{hawley1990artificial} & NN
    compared with probit & Artificial NN has a better performance \\
    \midrule
    Portfolio optimization & \cite{wong1998neural} & B\&H strategy &
    Artificial NN has a better performance \\
    \midrule
    Portfolio optimization & \cite{holsapple1988adapting} &
    Back-Propagation NN with mean-variance model & Artificial NN has a
    better performance \\
    \bottomrule
  \end{tabularx}
  \caption{Comparison between ANNs and other methods of portfolio
    management in \cite{beiranvand_comparative_2012}}
  \label{tab:ann-comparison-portfolio-manage}
\end{table}

\begin{table}[htbp]
  \scriptsize
  \myfloatalign
  \begin{tabularx}{\textwidth}{XXXX} 
    \toprule
    \tableheadline{Domain} & \tableheadline{Author(s)} &
    \tableheadline{Approaches compared} & \tableheadline{Conclusion} \\ 
    \midrule
    Selecting financial variables & \cite{markowitz1959portfolio} &
    ANN models that uses constant relevant variables compared to LR,
    B\&H strategy, and random walk & Artificial NN has a better
    performance \\
    \midrule
    Exchange rates prediction & \cite{lam2004neural} & ANN compared to
    linear autorefressive and random walk models & PNN and MLP have a
    better performance \\
    \midrule
    Post-bankruptcy resolutions & \cite{atsalakis2009surveying} & 5
    various models of ANN & Artificial NN has a better performance \\
    \midrule
    Bankruptcy forecasting & \cite{mochon2008soft} & ANN with MDA &
    Artificial NN has a better performance \\
    \midrule
    Classification of financial data &
    \cite{sivanandam2007introduction} & ANN with LDA and LR &
    Artificial NN has a better performance \\
    \midrule
    Asset value prediction & \cite{prasad2008soft} & BPNN with
    regression models & Artificial NN has a better performance \\
    \midrule
    Consumer price index forecasting & \cite{atiya2001bankruptcy} &
    ANN with random walk model & Artificial NN has a better
    performance \\
    \bottomrule
  \end{tabularx}
  \caption{Comparison between ANNs and other methods of forecasting
    and planning in \cite{beiranvand_comparative_2012}}
  \label{tab:ann-comparison-forecasting}
\end{table}

As for GAs and PSOs there aren't as much evidence of their suitability
for financial domain apart from this quote: \textit{``Mahfoud and Mani
  address the general problem of predicting future performances of
  individual stocks. They compare a genetic-algorithm-based system to
  an established neural network system on roughly 5000
  stock-prediction experiments. According to their experiments, both
  systems significantly outperform the B\&H strategy''}

Their conclusion is that \textit{``the performance and accuracy of the
  above mentioned artificial intelligence techniques are considerably
  higher, compared to the traditional statistical techniques,
  particularly in nonlinear models. Nevertheless, this superiority is
  not true in all cases.''}

%---------------------------------------------------------------------

\section[\cite{verikas2010hybrid}]{Hybrid and ensemble-based soft
  computing techniques in bankruptcy prediction: a survey}
\label{sec:hybrid-and-ensemble-based}

\cite{verikas2010hybrid} presents a comprehensive review of hybrid and
ensemble-based soft computing techniques applied to bankruptcy. The
focus is on techniques and not on obtained results, given that almost
all authors demonstrate that the technique they propose outperforms
some other methods chosen for the comparison.

This paper presents some techniques, which are shown in
\autoref{tab:hybrid-and-ensemble-based}, and then they draw some
conclusions. They mention that \textit{``fair comparison of results
  obtained in the different studies is hardly possible''} due to the
different dataset used, and different sizes of these datasets.
\textit{``Nonetheless the difficulty in comparing the reviewed
  techniques, one can make a general observation that ensembles, when
  properly designed, are more accurate than the other techniques. This
  is expected, since an ensemble integrates several predictors. In a
  successful ensemble design, a tradeoff between the ensemble accuracy
  and the diversity of ensemble members is achieved. [...]. However,
  transparency of ensemble-based techniques is rather limited, when
  compared to RS or IF-THEN rules- based approaches.''}

\begin{table}[htbp]
  \tiny
  \myfloatalign
  \begin{tabu} to \textwidth {X[1,1]
                              X[2,1]} 
    \toprule \tableheadline{Techniques} & \tableheadline{Model
      designing
      issues} \\
    \midrule
    \multicolumn{2}{c}{\textbf{Hybrid techniques}}\\
    GA + MLP & GA used for: finding MLP weights
    (\cite{pendharkar2004empirical, sai2007hybrid}); selecting
    features (\cite{back1996neural}); selecting features and MLP
    topology (\cite{ignizio1996simultaneous,
      wallrafen1996genetically}); selecting features then MLP topology
    (\cite{abdelwahed2005new}); finding weights of the fuzzy
    integral-based MLP (\cite{hu2008incorporating,
      hu2007functional})\\
    GA + SVM & GA used for: selecting SVM hyper-parameters
    (\cite{chen2008feature, wu2007real}); selecting features and SVM
    hyper-parameters (\cite{min2006hybrid, ahn2006global}).\\

    GA + k-NN & GA used for finding the appropriate k value (\cite{quintana2008early}).\\

    GA + NLN & GA used for finding the network topology and parameters
    (\cite{tsakonas2006bankruptcy}).\\

    RS + SVM & RS used for selecting features (\cite{zhou2007predicting}).\\

    RS + MLP & RS used for selecting features for MLP and generating
    decision rules (\cite{ahn2000integrated}).\\

    RS + GA & RS used to select features for algebraic expressions
    obtained by GA (\cite{mckee2002genetic}).\\

    RS + k-NN & RS used for estimating the class membership in a fuzzy
    k-NN classifier (\cite{bian2003fuzzy}). \\

    FS + MLP + GA & FS used to extract rules from MLP, which are then
    optimized by GA (\cite{lu2006explanatory}). \\
    
    FS + MOCO & FS used to extract rules, which are then optimized by
    MOCO (\cite{kumar2006bankruptcy}). \\

    FS + IL & FS and IL are combined to obtain a fuzzy decision tree
    (\cite{jeng1997film}).\\

    FS + MLP & FS and MLP are combined to obtain a neuro-fuzzy
    classifier (\cite{gorzalczany1999neuro}).\\

    FS + BP & FS used to design a fuzzy-neural network trained by BP.
    The initial structure of the network is obtained by
    self-organization (\cite{lee2006brain, ang2003popfnn, tung2004genso}).\\

    SOM + MLP & Prediction results obtained from MLP are superimposed
    on SOM visualizing the financial data (\cite{sun2008listed}); SOM is
    trained using the financial data augmented with the prediction
    obtained from the MLP (\cite{huysmans2006failure}).\\

    NT + MLP Feature set for MLP is augmented with Mahalanobis
    distance measures (\cite{markham1995combining}; features are
    transformed by applying NT (\cite{piramuthu1998using}).\\

    LDA + MLP & Features are selected by LDA ands then used in MLP
    (\cite{lee1996hybrid, back1996neural}); LDA is used to select
    features and also to generate an additional input to MLP
    (\cite{lee2002credit}).\\

% LR + MLP Features are selected by LR and then used in MLP (Back et al. 1996).
% LR + FR LR and FR are combined. Fuzzy regression parameters are found by solving a linear
% programming problem (Tseng and Lin 2005).
% CBR and IR techniques are combined (Elhadi 2000).
% CBR + IR
% Ensemble techniques
% MLP + k-NN + C4.5
% MLP, k-NN, and C4.5 are aggregated by stacking via a new MLP. To promote diversity,
% different techniques are used to select features for ensemble members (Shin et al. 2006).
% MLP Ensemble of 30 bagged MLPs (Shin et al. 2006).
% MLP Three different types of ensembles of 100 MLPs created using bagging, cross validation, or
% AdaBoosting (West et al. 2005).
% MLP To reduce dimensionality input features are transformed by applying PCA (Shin and Kilic
% 2006).
% RBF A bagged ensemble. Features are selected separately for each network trained on a separate
% bagged data set (Chan et al. 2006).
% RBF Diversity is promoted during the GA-based feature selection by including a diversity term in
% the fitness function. Features are selected simultaneously for all members (Yeung et al.
% 2007a, b).
% DT An ensemble of DT created using the AdaBoost algorithm (Alfaro et al. 2008; Cortes et al.
% 2007).
% SVM Bagged ensemble of SVMs, where different ensemble members use different hyper-
% parameters (Yu et al. 2007).
% MLP+LDA+LR+ MARS+C4.5 Aggregation by voting and weighted sum are explored. GA is used to find the aggregation
% weights (Olmeda and Fernandez 1997).

    \bottomrule
  \end{tabu}
  \caption{A survey of hybrid and ensemble-based soft computing
    techniques applied to predict bankruptcy by \cite{verikas2010hybrid}}
  \label{tab:hybrid-and-ensemble-based}
\end{table}

When referening to feature set used for predicting bankruptcy, they
mention: \textit{``Different studies indicate that the bankruptcy
  prediction accuracy can be increased substantially by including
  non-financial features into the modeling process and a trend is on
  using non-financial features, for example macroeconomic indicators
  and qualitative variables, in addition to financial ratios. [...].
  Genetic algorithms and RS are the two most popular approaches to
  feature selection in hybrid and ensemble-based techniques for
  bankruptcy prediction.''}

As a major downside of soft computing thechniques their conclusion is:
\textit{``The non-linear nature of hybrid and ensemble-based models
  and the lack of a widely accepted procedures for designing such
  models are major factors contributing to pitfalls in applications of
  these technologies. Model building, model selection and comparison
  are the designing steps where the most common pitfalls occur due to
  small sample sizes, model over-fitting or under-fitting, the
  sensitivity of solutions to initial conditions. The problem of
  selecting a model of appropriate complexity is often forgotten when
  developing soft computing techniques for bankruptcy prediction.''}

%---------------------------------------------------------------------
%---------------------------------------------------------------------
%---------------------------------------------------------------------

%\enlargethispage{2cm}

%------------------------------------------------

%%% Local Variables:
%%% mode: latex
%%% TeX-master: "../main"
%%% End:
