% Discussion

\chapter{Discussion}
\label{ch:discussion}

In this project we had the purpose of comparing two predictive models,
i.e. \textit{vector autoregression (VAR)} traditionally used in
econometric and \textit{recurrent neural network (RNN)} developed in
the area of \textit{artificial intelligence} that is increasingly been
used in a wider spectrum of fields. The two prediction techniques have
been applied to the one day ahead prediction of the \textit{Bitcoin}
price to measure their performance.

Once we have finished the experimentation we observe that the behavior
of \textit{VAR} in pronounced changes of the \textit{Bitcoin}'s
\textit{market price} is similar to that of \textit{RNN}. Although
\textit{RNN} compares better to \textit{VAR} when fed with more
elements of the \textit{time-series}.

Is worth mentioning that \textit{RNN} improves significantly as the
number of inputs neurons increases in such a way that some error
measures as \textit{mean squared error (MSE)} and \textit{Theil's $U$
statistic} move from performing worse than \textit{VAR} to performing
better.

\textit{RNN} are black-box models. There is no explicit form to
explain and analyze the relationship between inputs and outputs. This
causes difficulty in interpreting results from the networks. Besides
there are no structured methods today to identify what network
structure can best approximate the function, mapping the inputs to
outputs. Hence, the tedious experiments and trial-and-error procedures
are often used.

The results obtained in the experiment support the intuition that
non-linear models perform better than linear ones in prediction. At
the same time linear models have less computational cost than
non-linear models, this is because the number of different functions
that can be represented by linear models is far less than the ones
that non-linear models can represent. In terms of computational cost
\textit{VAR} performs approximately 10 times better

As to the features that better predict the \textit{Bitcoin}'s price,
we observed that the only two variables that where selected by the
different \textit{feature subset selection} techniques and models
where \textit{total bitcoins} and \textit{market price}. It has to be
taken into account that \textit{wrapper feature subset selection
(WFSS)} is an heuristic technique that doesn't guarantee the optimum
subset of features because the algorithm can get stuck in local
optima.

Possible ideas that arise from this study and the experiments that we
have run would be:

\begin{itemize}
\item Include new features, trying to explore new currencies or maybe
  include social media information through sentiment analysis.
\item Another approach to obtain better performance results can be
  reached by using different variants of the models used, such as
  \textit{vector autoregressive moving-average models (VARMA)} or
  \textit{vector autoregressive integrated moving average (VARIMA)}
  variants of \textit{VAR}, or \textit{long short-term memory (LSTM)}
  and \textit{gated recurrent units (GRU)} variants of \textit{VAR}.
  Even new models can be tested in order to check how they predict the
  \textit{Bitcoin} price.
\end{itemize}

%---------------------------------------------------------------------
%---------------------------------------------------------------------
%---------------------------------------------------------------------

%\enlargethispage{2cm}

%------------------------------------------------

%%% Local Variables:
%%% mode: latex
%%% TeX-master: "../main"
%%% End:
